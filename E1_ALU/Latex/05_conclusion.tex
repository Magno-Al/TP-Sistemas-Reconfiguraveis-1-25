\section{Conclusão}

O desenvolvimento de uma Unidade Lógica e Aritmética (ALU) de 8 bits utilizando a linguagem VHDL permitiu consolidar conhecimentos teóricos e práticos relacionados à lógica digital, à modelagem de hardware e ao uso do ambiente de desenvolvimento Quartus II.

A implementação foi conduzida de forma estruturada, respeitando os requisitos propostos, com a construção de um circuito totalmente combinacional e a aplicação de técnicas de codificação concorrente. A ALU resultante foi capaz de executar corretamente as 16 operações especificadas, incluindo funções aritméticas, lógicas, rotações e deslocamentos, além de sinalizar adequadamente os estados de carry, zero e overflow.

As simulações realizadas confirmaram a conformidade do comportamento funcional do projeto com as especificações fornecidas. O uso de arquivos de forma de onda (\texttt{.vwf}) e os relatórios de simulação evidenciaram que todas as operações foram validadas com sucesso.